
% This LaTeX was auto-generated from MATLAB code.
% To make changes, update the MATLAB code and republish this document.

\documentclass{article}
\usepackage{graphicx}
\usepackage{color}

\sloppy
\definecolor{lightgray}{gray}{0.5}
\setlength{\parindent}{0pt}

\begin{document}

    
    
\subsection*{Contents}

\begin{itemize}
\setlength{\itemsep}{-1ex}
   \item Problemas de flujo interno
   \item Problema 68
   \item Problema 69
\end{itemize}


\subsection*{Problemas de flujo interno}

\begin{verbatim}
clc
g=9.8065;
\end{verbatim}


\subsection*{Problema 68}

\begin{par}
Ventilador de una mina. Encontrar el incremento de presi�n que debe producir el ventilador.
\end{par} \vspace{1em}
\begin{par}
\textbf{Recuerden:}
\end{par} \vspace{1em}
\begin{itemize}
\setlength{\itemsep}{-1ex}
   \item que deben considerar la presi�n a la salida como \textit{Pamb} m�s la presi�n de columna de fluido, en este caso aire.
   \item el punto inicial para plantear la ecuaci�n de la energ�a est� ubicado por fuera del \textit{bell mouth}.
\end{itemize}
\begin{par}
Datos:
\end{par} \vspace{1em}
\begin{verbatim}
h= 700; rho = 1.225;D = 0.5;mu = 1.78E-5;
L = 700 + 20; K = 3.5; L_cod = 12; eta = .75;
p1=0; p2=rho*g*h;z1=0;z2=700;Q=120/60;%m3/seg
A=pi*D^2/4;
V=Q/A;
nu = mu/rho;
epsilon=0.15E-3; % chapa galvanizada
Re=V*D/nu;
% factor de fricci�n
 f=f_SJ(V,mu/rho,D,epsilon);
\end{verbatim}

        \color{lightgray} \begin{verbatim}
f0 =

    16.8266e-003


f =

    16.7309e-003

\end{verbatim} \color{black}
    \begin{par}
$$ h_{bomba}= \frac{V^2}{(2*g)}\left[1+\frac{f}{D}(L + 2 L_{cod})+K)\right]$$
\end{par} \vspace{1em}
\begin{verbatim}
h_bomba= V^2/(2*g)*(1+(L + 2*L_cod)*f/D+K);
Dp_bom = h_bomba*rho*g
P_bom = h_bomba*rho*g*Q/eta
P_bom_hp = P_bom/745.7
\end{verbatim}

        \color{lightgray} \begin{verbatim}
Dp_bom =

     1.8680e+003


P_bom =

     4.9815e+003


P_bom_hp =

     6.6802e+000

\end{verbatim} \color{black}
    \begin{par}
los resultados de $\Delta p_{bomba}$ y la potencia son similares a la guia de ejercicios
\end{par} \vspace{1em}


\subsection*{Problema 69}

\begin{par}
Ducto de ventilaci�n, determinar la potencia del ventilador seg�n:
\end{par} \vspace{1em}
\begin{enumerate}
\setlength{\itemsep}{-1ex}
   \item Secci�n de conducto circular
   \item Secci�n de conducto rectangular
\end{enumerate}
\begin{verbatim}
clear
rho = 1.225;D = 0.32;mu = 1.78E-5; nu = mu/rho;
rhoH2O = 1000; hH2O = .01;
L = 100; eta = .5; g=9.8065;
p1=0; p2=rhoH2O*g*hH2O
Q = 0.8;%m3/seg
\end{verbatim}

        \color{lightgray} \begin{verbatim}
p2 =

    98.0650e+000

\end{verbatim} \color{black}
    \begin{par}
opcion 1)
\end{par} \vspace{1em}
\begin{verbatim}
    A=pi*D^2/4; V = Q/A;
    epsilon=0.15E-3;% chapa galvanizada
    Re=V*D/nu;
    % factor de friccion
    f=f_SJ(V,mu/rho,D,epsilon);
    h_l = f*L/D*V^2/2
    h_bomba= p2/(rho*g)+V^2/(2*g)*(1+f*L/D);
    Dp_bom = h_bomba*rho*g;
    P_bom = h_bomba*rho*g*Q/eta
    P_bom_hp = P_bom/745.7
\end{verbatim}

        \color{lightgray} \begin{verbatim}
f0 =

    18.6127e-003


f =

    18.4968e-003


h_l =

   285.9681e+000


P_bom =

   814.3690e+000


P_bom_hp =

     1.0921e+000

\end{verbatim} \color{black}
    \begin{par}
opcion 2) El di�metro hidr�ulico se despeja de hacer iguales las �reas,
\end{par} \vspace{1em}
\begin{par}
\textit{D\_e} es el efectivo
\end{par} \vspace{1em}
\begin{verbatim}
    D_h = 4/3*D*sqrt(pi/8)
    D_e = 0.65 * D_h
    % factor de friccion
    f=f_SJ(V,mu/rho,D_e,epsilon);
    h_l = f*L/D_h*V^2/2
    h_bomba= p2/(rho*g)+V^2/(2*g)*(1+f*L/D_h);
    Dp_bom = h_bomba*rho*g;
    P_bom = h_bomba*rho*g*Q/eta
    P_bom_hp = P_bom/745.7
\end{verbatim}

        \color{lightgray} \begin{verbatim}
D_h =

   267.3737e-003


D_e =

   173.7929e-003


f0 =

    21.4921e-003


f =

    21.3310e-003


h_l =

   394.6959e+000


P_bom =

     1.0275e+003


P_bom_hp =

     1.3779e+000

\end{verbatim} \color{black}
    \begin{par}
Para los problemas este resultado es m�s preciso.
\end{par} \vspace{1em}
\begin{par}
\textbf{El resultado de la guia de ejercicios est� calculado sin usar la correcci�n de Jones.}
\end{par} \vspace{1em}



\end{document}
    
